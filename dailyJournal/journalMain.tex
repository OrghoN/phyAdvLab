\documentclass[oneside]{tufte-book}
% \documentclass[12pt, oneside]{article}

\title{Daily Journal\\Muon Detection}
\author{Orgho Neogi}

\usepackage{amssymb}
\usepackage{amsmath}
\makeatletter

\usepackage{graphicx}

\usepackage{float}

% \usepackage[colorlinks=true]{hyperref}

\usepackage{enumitem}

\usepackage[english]{babel}

\usepackage{amsthm,thmtools}
\usepackage[nottoc]{tocbibind}

\usepackage{caption}

\usepackage{longtable}

\usepackage{lipsum}

\newenvironment{loggentry}[2]% date, heading
% {\noindent\textbf{#2}\marginnote{#1}\\}{\vspace{0.5cm}}
{\noindent\huge{\textbf{#2}}\normalsize\vspace{0.5cm}\marginnote{#1}\\}{\vspace{0.5cm}}

\setlength{\parskip}{1em}
\usepackage{geometry}

\geometry{
  left=1in, % left margin
  textwidth=35pc, % main text block
  marginparsep=2pc, % gutter between main text block and margin notes
  marginparwidth=5pc % width of margin notes
}


\begin{document}

  \maketitle

  % \tableofcontents

  \begin{loggentry}{2018-Aug-27}{High Voltage Power}

    Geiger tubes need high voltage power (400V) and to produce it, we are using a 16EPC-T06 module .

    Parts List:
    \begin{enumerate}
      \item 16EPC-T06 (High Voltage Generator)
      \item 100nF capacitor
      \item LM317 (Voltage Regulator)
      \item 10K POT
      \item 470 $\Omega$ resistor
      \item 1000 nF capacitor
    \end{enumerate}

  \end{loggentry}

  \begin{loggentry}{2018-Aug-28}{Bridge Rectifier}

    Got high voltage AC output out of the 16EPC-T06 at 400 V. Finished making the bridge rectifier afterwards, but it is only doing half wave rectification. Probably something to do with how the circuit is wired since the diodes are working fine individually. Tested it out with a signal generator and scope, still doing half wave rectification so not a problem with high voltage circuit.

    Multimeter cannot read high voltage AC since it seems the frequency is too high but the scope with a voltage divider can read it fine.

  \end{loggentry}

  \begin{loggentry}{2018-Aug-29}{True Ground}

      The issue with the rectifier was not in measuring the circuit but rather in how the measurement was happening.

    \begin{figure}
        \centering
        % \def\svgscale{0.2}
        \input{correctBridgeRectifier.pdf_tex}
        \caption{Bridge Rectifier Circuit}
        \label{correctBridgeRectifier}
    \end{figure}

      The correct bridge rectifier is Figure \ref{correctBridgeRectifier} and I had built the circuit correctly but there was a problem in measuring with the scope since I had hooked up the scope ground to $V_-$.

      This situation is represented by Figure \ref{bridgeRectifierProblem}.

      \begin{figure}
          \centering
          \input{bridgeRectifierProblem.pdf_tex}
          \caption{Measurement Problem. Red represents current flow}
          \label{bridgeRectifierProblem}
      \end{figure}

      When attempting to measure, current would flow through when the input voltage was positive but no current would flow when it was negative thus resulting in the half wave rectification we saw on the scope.

      To measure correctly, one channel of the scope has to be hooked up to $V_+$ and ground while the other is hooked to $V_-$ and ground. The two channels can then be subtracted to measure the fully rectified signal.

      \begin{figure}
          \centering
          \includegraphics[width=90mm]{rectifierScope.jpg}
          \caption{Scope output of rectified signal}
          \label{rectifierScope}
      \end{figure}

      Parts List:
      \begin{enumerate}
        \item diode $\times$ 4
        \item 100 $\mu$F 400 V capacitor
      \end{enumerate}

      \textbf{Note:} If an electrolytic capacitor has been sitting out for a long time, before applying high voltage, fill it up for a while (an hour should be good) with low voltage (12V). That allows it to fix internal issues or fail in a not so destructive fashion.

  \end{loggentry}

  \begin{loggentry}{2018-Aug-30}{Single Geiger Readout}
    There is an issue with the high voltage power supply. Instead of outputting 400V from 1.6V as it was 2 days ago, now it is giving an output of around 110V. The circuit has been isolated and it is receiving 1.6V.

    Changing the input voltage to the high voltage power supply to 2.57V fixes the output of the high voltage supply after rectification to 395.2V.

    Parts List:
    \begin{enumerate}
      \item diode $\times$ 4
      \item 100 $\mu$F 400 V capacitor
    \end{enumerate}

    \begin{figure}
        \includegraphics[width=\linewidth]{highVoltagePowerEC.jpg}
        \caption{High Voltage Power supply for Geiger counter}
        \label{powerSupplyReal}
    \end{figure}

    To measure high voltage with scope, a voltage divider needs to be used. Since the scope will be plugged into V$_+$ and V$_-$ as well as ground, both voltages need to be divided.

    \begin{figure}
        \centering
        \input{scopeMeasurement.pdf_tex}
        \caption{Measuring High Voltage With Scope from Rectifier. The resistor values attenuate signal $\times$100}
        \label{scopeMeasurement}
    \end{figure}

    \begin{enumerate}
      \item 10 M$\Omega$ resistor
      \item 1K$\Omega$ resistor
      \item 1M$\Omega$ resistor
      \item diode $\times$ 2
      \item 5.6 pF capacitor
      \item 1nF capacitor
      \item 100 nF capacitor
      \item 74HC04 hex NOT gate
    \end{enumerate}

    A single geiger tube, once hooked up is producing pulses at an appropriate rate.

    \begin{figure}
        \centering
        \includegraphics[width=90mm]{geigerPulse.jpg}
        \caption{Negative pulse from single geiger tube}
        \label{geigerScope}
    \end{figure}
  \end{loggentry}

  \begin{loggentry}{2018-Aug-31}{Soldering down Power}
    Soldered down high voltage power supply into breadboard. The output voltage is 380 V.

    \begin{figure}
        \centering
        \includegraphics[width=45mm]{highVoltagePowerBB.jpg}
        \caption{High Voltage power supply taking in 5V and Ground on red and black wires and outputting on orange and gray ones.}
        \label{highVoltagePowerSolder}
    \end{figure}

    Scope doesn't explain why 400V was not found on the output end because the inital breadboard gave 395 V.
  \end{loggentry}


\end{document}
